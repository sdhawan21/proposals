\documentclass{article}
\begin{document}
\title{Science Case for SN2014J +300d proposal for NIR Spectra}
\maketitle

SN2014J is a nearby supernova of Type Ia in the spiral galaxy M82. It is heavily occluded by host galaxy dust (E(B-V)=1.3,  Patat+ 2014). It also exhibits a peculiar reddening law (Amanullah+2014). This makes it difficult to extract useful information from optical only observations and hence, makes observations in the NIR imperative. 

From $\gamma$-ray observations, there has been a direct detection of Cobalt at late times in the ejecta (Churazov+2014, Diehl+2014b). This offers an extinction-independent method of deriving the total amount of $^{56}$Ni produced in the SN.An NIR spectrum at $\sim$ +300 days can provide a firm constraint on the extinction value, which along with the multi-band photometry available (eg. Foley+2014)can be used to derive an $^{56}$ measurement that is independent of the value $\gamma$ ray observations.   


\section{Nebular epoch velocity effects (to revise title)}
In Mazzali+1998 and Blondin+2012 (Cfa+Mazzali sample), the authors show a relation between the nebular FWHM of the FeIII4700 line and Dm15, although in the latter reference, there is less of  a claim for a correlation in the 'normal' Ia sample and the authors claim that this correlation is driven by 86G and 91bg.

In S04, there are 2 NIR spectra from Sofi and ISAAC at +250 and +344 days. The authors observe a shift in the velocity of the FeIII lines. They argue that the relation for nebular spectra can then be a function of the epoch of observations.

Getting a spectral series of 4-5 spectra interspersed at 50d intervals might be important for observing this effect

\section{Asymmetries}


\section{Extinction Measurement}
SN2014J is a heavily reddened supernova with an optical extinction (E(B-V)) of 1.3 of which1.16 mags is contributino from the host galaxy. It also has a peculiar reddening law which makes the final, derived absorption vlaue in the optical, very highly debatable. In Margutti+2014, the Av is 1.7 mag and in goobar+ it is 2.5 mag. this yields remarkably different $M_{Ni}$ values from the bolometric light curve of 0.37 and 0.77 $M_{\odot}$. In order to resolve this discrepancy, an NIR spectrum will be useful. The ratio of line widths from the 1.644 $\mu m$ and 1.275 $\mu m$ would yield an absorption value that can be applied to the bolometric light curve calculated from existing data. 

The proximity of SN2014J has led to the first direct detection of $^{56}Co$ at late epochs in the ejecta. This measurement of Ni mass can be compared with the value derived using the extinction measurement from the NIR spectrum

 



\end{document}





