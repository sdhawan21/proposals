\documentclass{article}
\begin{document}
\title{Science Case: IFU spectroscopy}
\maketitle
Type Ia supernovae are exceptional distance indicators for measurement of cosmological parameters. Their use as 'standardized' candles led to the discovery of dark energy. This was possible, by calibrating the peak lumniosity using strong correlations with the post-peak decline rate and the optical colour. The width-luminosity relation is shown to be a  result of a variance in the total $^{56}Ni$ synthesized in the SN ejecta. However, the scatter in the width-luminosity relation has evoked the need for a secondary parameter to explain SNIa light curves. On possibility for this is the metallicity of the progenitor, and as a result, the environment of the SN in the host. 

Studies in the past have found correlations between host properties  (eg. stellar mass) and supernova peak luminosities and optical colour (Sullivan et al. [2010]).  With the SNfactory sample have demonstrated a correlation between the host $M_{stellar}$, $sSFR$, gas-phase metallicities and the SNIa hubble residual (Childress et al. [2013]).
This indicates that bluer, brighter SN occur more frequently in younger stellar populations. Theoretical efforts (eg. Timmes et al. [2003]) have shown that higher metallicity in the environments leads to more neutron rich nuclear species and hence a lower production of 
$^{56}Ni$, which qualitatively explains the observed trend between the SN intrinsic brightness and the metallicity of the host. 

In this project, we aim to measure the metallicities for SN hosts with more than one SN. The targets host both 'normal' Ia's as well as peculiar, subluminous objects. A comparison of the different regions in these hosts will allow us to link the properties of different SN subclasses to the environments in which they explode. 

Candidate objects in our sampel have excellent complementary datasets for the hosts, as well as the SNe. NGC 1309 has precise distance measurements from Cepheids (Riess et al. [2011]). It hosted two
SNe, a well studied, normal SN, 2002fk and an interesting, peculiar Type Iax explosion SN2012Z. These are characteristically subluminous explosions, with very little production of $^{56}Ni$ . From HST pre-explosion imaging, Mccully et al. [2014] have, for the first time identified the progenitor star of such an explosion. Hence, NGC 1309 presents the ideal opportunity to study the difference between the environments of normal SNIa and more peculiar Iax SNe. 

%NGC 3109 hosted two SNIa 2002bo and 2002cv. 2002cv is an extremely extinguished SN due to host galaxy dust. It wasn't visible in the B band. A measurement of the extinction from the balmer decrement will allow us to understand the dust in the region surrounding the SN.
\end{document}
