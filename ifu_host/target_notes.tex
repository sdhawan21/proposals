\documentclass{article}

\begin{document}
\title{IFU spectroscopy of multiple SN host}
\maketitle

\section{Science Case}
Studies of type Ia supernova (SN Ia) hosts have shown remarkable correlations between host galaxy properties and the luminosity of the supernova (Sullivan+2006, 2010). Recent studies (Pan+ 2014) have demonstrated a direct relation between host galaxy stellar mass, specific star formation rate (sSFR) and the spectral properties of SNIa. This provides us with a different angle to study the nature of SN explosions.

In this study we propose to understand the relation between host galaxy metallicity and photometric and spectroscopic properties using integral field spectroscopy of 3 prolific SNIa hosts. As a consequence, we would also measure the extinction using Balmer decrement and compare to other methods with published results. 

the galaxy NGC 1316 has produced 4 SNIa in the last $\sim$ 30 years. 3 of these are classified as 'Branch normal' SNe while SN2006mr is a faint, SN1991bg-like explosion. This offers us an opportunity to observe differences in the explosion environments of faint Ia's from normal Ia's.

A similar comparison can also be made for the two SNe in NGC1309. SN2002fk is a normal Ia with extensive photometry and a spectral time series, where SN2012Z is a known underluminous, peculiar Ia. The recent discovery of a progenitor system for SN2012Z (Mccully+2014) allows us to tie the progenitor systems to the environments of the peculiar Type Iax explosions. 



 
\section{Target List}
For this proposal, we aim to get IFU spectroscopy of galaxies which host more than one SNIa

Current target list:



I:NGC 1316: 2006mr, 2006dd, 1981D, 1980N

RA: 03h22m41.7s; DEC:-37d12m30s

Major diameter:	8.5
Minor diameter:	7.5 (from ESO-Uppsala)

1. Stritzinger+ has measured distances to the galaxy using the different SNIa's 
All of them have well-sampled light curves and 3 have a spectral time series.



II:NGC 1309: 2012Z (Iax), 2002fk

RA:03h22m06.5s; DEC:-15d24m00s


Major diameter:	2.2
Minor diameter:	2.0

1. 2012Z has a progenitor identified. (Mccully+)
2. 2002fk has multiband photometry and spectroscopy at pre- to post- maximum epochs
(cartier+)
There is a Cepheid distance from the HST $3\%$ solution paper by A. Riess


III: NGC 3190: 2002bo, 2002cv

RA:10h18m05.6s	; DEC: +21d49m56s
Major diameter : 4.4'
Minor  " "    : 1.5'

1. 2002bo has been observed by Pastorello+. Modelling efforts by Stephane are in prep for the SN, looking at NIR spectra.

2. 2002cv is heavily occluded and invisible in B band. Looking at the region might be interesting for better understanding of dust.


\end{document}
