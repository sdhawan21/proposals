%%%%%%%%%%%%%%%%%%%%%%%%%%%%%%%%%%%%%%%%%%%%%%%%%%%%%%%%%%%%%%%%%%%%%%
%
%.IDENTIFICATION $Id: template.tex.src,v 1.41 2008/01/25 10:47:12 fsogni Exp $
%.LANGUAGE       TeX, LaTeX
%.ENVIRONMENT    ESOFORM
%.PURPOSE        Template application form for ESO Observing time.
%.AUTHOR         The Esoform Package is maintained by the Observing
%                Programmes Office (OPO) while the background software
%                is provided by the User Support System (USS) Department.
%
%-----------------------------------------------------------------------
%
%
%                   ESO LA SILLA PARANAL OBSERVATORY
%                   --------------------------------
%                   NORMAL PROGRAMME PHASE 1 TEMPLATE
%                   ---------------------------------
%
%
%
%          PLEASE CHECK THE ESOFORM USERS' MANUAL FOR DETAILED 
%              INFORMATION AND DESCRIPTIONS OF THE MACROS. 
%     (see the file usersmanual.tex provided in the ESOFORM package) 
%
%
%        ====>>>> TO BE SUBMITTED THROUGH WEB UPLOAD  <<<<====
%               (see the Call for Proposals for details)
%
%%%%%%%%%%%%%%%%%%%%%%%%%%%%%%%%%%%%%%%%%%%%%%%%%%%%%%%%%%%%%%%%%%%%%%

%%%%%%%%%%%%%%%%%%%%%%%%%%%%%%%%%%%%%%%%%%%%%%%%%%%%%%%%%%%%%%%%%%%%%%
%
%                      I M P O R T A N T    N O T E
%                      ----------------------------
%
% By submitting this proposal, the Principal Investigator takes full
% responsibility for the content of the proposal, in particular with
% regard to the names of CoI's and the agreement to act in accordance
% with the ESO policy and regulations, should observing time be
% granted.
%
%%%%%%%%%%%%%%%%%%%%%%%%%%%%%%%%%%%%%%%%%%%%%%%%%%%%%%%%%%%%%%%%%%%%%% 

%
%    - LaTeX *is* sensitive towards upper and lower case letters.
%    - Everything after a `%' character is taken as comments.
%    - DO NOT CHANGE ANY OF THE MACRO NAMES (words beginning with `\')
%    - DO NOT INSERT ANY TEXT OUTSIDE THE PROVIDED MACROS
%

%
%    - All parameters are checked at the verification or submission.
%    - Some parameters are also checked during the pdfLaTeX
%      compilation.  If this is not the case, this is indicated by the
%      phrase
%      "This parameter is NOT checked at the pdfLaTeX compilation."
%

\documentclass{esoform}

% The list of LaTeX definitions of commonly used astronomical symbols
% is already included in the style file common2e.sty (see Table 1 in
% the Users' Manual).  If you have your own macros or definitions,
% please insert them here, between the \documentclass{esoform}
% and the \begin{document} commands.
%
%     PLEASE USE NEITHER YOUR OWN MACROS NOR ANY TEX/LATEX MACROS  
%       IN THE \Title, \Abstract, \PI, \CoI, and \Target MACROS.
%
% WARNING: IT IS THE RESPONSIBILITY OF THE APPLICANTS TO STAY WITHIN THE
% CURRENT BOX LIMITS AND ELIMINATE POTENTIAL OVERFILL/OVERWRITE PROBLEMS 

\begin{document}

%%%%%%%%%%%%%%%%%%%%%%%%%%%%%%%%%%%%%%%%%%%%%%%%%%%%%%%%%%%%%%%%%%%%%%%%
%%%%% CONTENTS OF THE FIRST PAGE %%%%%%%%%%%%%%%%%%%%%%%%%%%%%%%%%%%%%%%
%%%%%%%%%%%%%%%%%%%%%%%%%%%%%%%%%%%%%%%%%%%%%%%%%%%%%%%%%%%%%%%%%%%%%%%%
%
%---- BOX 1 ------------------------------------------------------------
%
% You should use this template for period 94A applications ONLY.
%
% DO NOT EDIT THE MACRO BELOW. 

\Cycle{94A}

% Type below, within the curly braces {}, the title of your observing
% programme (up to two lines).
% This parameter is NOT checked at the pdfLaTeX compilation.
%
% DO NOT USE ANY TEX/LATEX MACROS IN THE TITLE

\Title{Measuring the orbital decay of the close sdB+WD binary CD$-$30\,11223 caused by gravitational wave emission}  

% Type below the numeric code corresponding to the subcategory of your
% programme.

\SubCategoryCode{D7}   

% Please specify the type of programme you are submitting. 
% Valid values: NORMAL, GTO, TOO, CALIBRATION, MONITORING
% If you specify TOO, you will also need to fill a ToO page below.
% If you specify CALIBRATION, then the SubCategory Code MUST be set to L0

% If your programme requires more than 100 hours the Large Programme
% template (templatelarge.tex) must be used.


\ProgrammeType{NORMAL}

% For GTO proposals only: uncomment the following and fill out the GTO
% programme code (as communicated to the respective GTO coordinator).

%\GTOcontract{INS-consortium}		

% For TOO proposals only: uncomment the following if you apply for
% Rapid Response Mode observations.
 
%\ObservationInRRM{}

% Uncomment the following macro if this proposal is applying for time
% under the VLT-XMM agreement (only available for odd periods).

%\ObservationWithXMM{}

%---- BOX 2 ------------------------------------------------------------
%
% Type below a concise abstract of your proposal (up to 9 lines).
% This parameter is NOT checked at the pdfLaTeX compilation.
%
% DO NOT USE ANY TEX/LATEX MACROS IN THE ABSTRACT

\Abstract{Short period binaries are predicted to be strong sources of gravitational waves. Here we propose to use time resolved FORS2 photometry of the bright, eclipsing, short-period sdB+WD binary CD$-$30\,11223, to constrain the value of the orbital decay $\dot{p}$, as indirect evidence of gravitational wave emission. We want to improve the SNR on the shallow eclipses and improve the orbital parameters determination of this unique binary. Combining the new, high-precision eclipse times with archive data, we will be able to detect the predicted shrinkage of the orbit.}

%---- BOX 3 ------------------------------------------------------------
%
% Description of the observing run(s) necessary for the completion of
% your programme.  The macro takes ten parameters: run ID, period,
% instrument, time requested, month preference, moon requirement,
% seeing requirement, transparency requirement, observing mode and 
% run type.
%
% 1. RUN ID
% Valid values: A, B, ..., Z
% Please note that only one run per intrument is allowed for APEX
%
% 2. PERIOD
% Valid values: 94
% Exceptions:
% Monitoring Programmes: These programmes can span up to four periods.
%
% VLT-XMM proposals: These are only accepted in odd periods and are 
% also valid for the next period.
%
% This parameter is NOT checked at the pdfLaTeX compilation.
%
% 3. INSTRUMENT
% Valid values: AMBER ARTEMIS CHAMPP EFOSC2 FLAMES FLASH FORS2 HARPS HAWKI KMOS LABOCA MIDI MUSE NACO OMEGACAM SABOCA SHFI SINFONI SOFI SOFOSC SUPERCAM Special3.6 SpecialAPEX SpecialNTT SpecialVLTI UVES VIMOS VIRCAM XSHOOTER
% 
% Only Chilean and GTO Programmes are accepted on OMEGACAM.
% No normal programmes on OMEGACAM will be accepted.
% Please note that only a subset of these instruments will be accepted
% for Monitoring Programmes. Please see the Call for Proposals and the
% ESOFORM User Manual for more details.
%
% 4. TIME REQUESTED
% In hours for Service Mode, in nights for Visitor Mode.
% In either case the time can be rounded up to  1 decimal place. 
% This parameter is NOT checked at the pdfLaTeX compilation.
% 
% 5. MONTH PREFERENCE
% Valid values: oct, nov, dec, jan, feb, mar, any
%
% 6. MOON REQUIREMENT
% Valid values: d, g, n
%
% 7. SEEING REQUIREMENT
% Valid values: 0.4, 0.6, 0.8, 1.0, 1.2, 1.4, n
%
% 8. TRANSPARENCY REQUIREMENT
% Valid values: CLR, PHO, THN
%
% 9. OBSERVING MODE
% Valid values: v, s
%
% 10. RUN TYPE
% Valid values: TOO 
% For all Normal & Calibration Programmes this field should be blank.
% For TOO & GTO Programmes, users can specify TOO runs.
% If the field is left blank a default normal, non-TOO run is assumed.
% If a TOO run is specified please make sure that you fill in the TOO page.



\ObservingRun{A}{94}{FORS2}{4h}{any}{n}{n}{THN}{s}{}
%\ObservingRun{A/alt}{94}{FORS2}{3n=2x1+2H2}{nov}{n}{0.8}{PHO}{v}{}
%\ObservingRun{B}{94}{VIMOS}{2n=2x1}{dec}{n}{0.6}{CLR}{v}{}
%\ObservingRun{C}{94}{EFOSC2}{3n}{feb}{n}{0.8}{THN}{v}{}
%\ObservingRun{D}{94}{NACO}{0.4n}{nov}{n}{0.8}{THN}{v}{}
%\ObservingRun{E}{94}{AMBER}{1h}{oct}{n}{1.4}{THN}{s}{}
%\ObservingRun{F}{94}{MIDI}{1h}{oct}{n}{n}{THN}{s}{}


% Proprietary time requested.
% Valid values: % 0, 1, 2, 6, 12

\ProprietaryTime{12}

%---- BOX 4 ------------------------------------------------------------
%
% Indicate below the telescope(s) and number of nights/hours already
% awarded to this programme, if any.
% This macro is optional and can be commented out.
% It is also NOT checked at the pdfLaTeX compilation.

%\AwardedNights{NTT}{4n in 92.B-1234}

% Indicate below the telescope(s) and number of nights/hours still
% necessary, in the future, to complete this programme, if any.
% This macro is optional and can be commented out.
% It is also NOT checked at the pdfLaTeX compilation.

%\FutureNights{UT2}{20h}

%---- BOX 5 ------------------------------------------------------------
%
% Take advantage of this box to provide any special remark  (up to three
% lines). In case of coordinated observations with XMM, please specify
% both the ESO period and the preferred month for the XMM
% observations here.
% This macro is optional and can be commented out.
% It is also NOT checked at the pdfLaTeX compilation.

%\SpecialRemarks{This macro is optional and can be commented out. }
  
%---- BOX 6 ------------------------------------------------------------
% Please provide the ESO User Portal username for the Principal
% Investigator (PI) in the \PI field.
%
% For the Co-I's (CoI) please fill in the following details:
% First and middle initials, family name, the institute code
% corresponding to their affiliation. 
% The corresponding affiliation should be entered for EACH
% Co-I separately in order to ensure the correct details of 
% all Co-I's are stored in the ESO database.
% You can find all institute codes listed according to country
% on the following webpage:
% http://www.eso.org/sci/observing/phase1/countryselect.html
%
% For example, if the Co-I's full name is David Alan William Jones,
% his affiliation is the Observatoire de Paris, Site de Paris, 
% you should write:
% \CoI{D.A.W.}{Jones}{1588}
% Further examples are shown below.
% DO NOT USE ANY TEX/LATEX MACROS HERE
%

\PI{sdhawan} 
% Replace with PI's ESO User Portal username.

\CoI{S.}{Geier}{1098}
\CoI{T.}{Kupfer}{1638}

% Please note: 
% Due to the way in which the proposal receiver system parses
% the CoI macro, the number of pairs of curly brackets '{}'
% in this macro MUST be strictly equal to 3, i.e., the
% number of parameters of the macro. Accordingly, curly
% brackets should not be used within the parameters (e.g.,
% to protect LaTeX signs).
%
% For instance:
% \CoI{L.}{Ma\c con}{1098}
% \CoI{R.}{Men\'endez}{1098}
%
% are valid, while
%
% \CoI{L.}{Ma{\c}con}{1098}
% \CoI{R.}{Men{\'}endez}{1098}
%
% are not. Unfortunately the receiver does not give an
% explicit error message when such invalid forms are
% used in the CoI macro, but the processing of the proposal
% keeps hanging indefinitely.


%%%%%%%%%%%%%%%%%%%%%%%%%%%%%%%%%%%%%%%%%%%%%%%%%%%%%%%%%%%%%%%%%%%%%%%%
%%%%% THE TWO PAGES OF THE SCIENTIFIC DESCRIPTION AND FIGURES %%%%%%%%%%
%%%%%%%%%%%%%%%%%&&&%%%%%%%%%%%%%%%%%%%%%%%%%%%%%%%%%%%%%%%%%%%%%%%%%%%%
%
%---- BOX 7 ------------------------------------------------------------
%
%               THIS DESCRIPTION IS RESTRICTED TO TWO PAGES 
%
%   THE RELATIVE LENGTHS OF EACH OF THE SECTIONS ARE VARIABLE,
%   BUT THEIR SUM (INCLUDING FIGURES & REFS.) IS RESTRICTED TO TWO PAGES
%
% All macros in this box are NOT checked at the pdfLaTeX compilation.

\ScientificRationale{The recent discovery of primordial gravitational waves (GW) in the B-mode polarization power spectrum of the CMB has led to a renewed interest in the hunt for gravitational waves. One of the possible, but yet unconfirmed, methods of detecting GW is through direct measurements from ground and spaced-based missions. A more indirect method for detecting the presence of gravitational waves is by using short-period compact binary systems. General relativity (GR) predicts that close binary systems will lose energy in the form of emission of GW. This leads to orbital shrinkage which can be inferred from the observables of the system, especially the change (shortening) of the orbital period. Such an inferrence was first made with the famous Hulse-Taylor binary pulsar (PSR\,B1913+16). Extensive observations over a period of decades confirmed the shift of the periastron (Hulse and Taylor [1975]). This remarkable discovery led to the authors getting the Nobel Prize in 1993. 

It was expected that such orbital decays should also be observable in close binaries with compact components other than pulsars. Especially, if such compact binaries should show eclipses. Eclipsing binaries are resourceful astrophysical laboratories. Precise measurements of the eclipses and radial velocities allow us to evaluate the masses and radii of the components. Furthermore, a measure of the change in period over time ($\dot{p}$) provides indirect evidence of gravitational waves from the shrinkage of the orbit. Since the range of inclinations for the eclipses to be observable is very small, such systems are very rare and we therefore require detailed studies of known objects. The recent discovery of a double white dwarf (WD) binary (Brown et al. [2011]) with a period of $765\,{\rm s}$ provided an independent opportunity to confirm the shortening of the period. Hermes et al. [2012] were able to put constraints on the orbital shrinkage of J0651. They measured a shift of $-9.8\pm2.8\times10^{-12}\,{\rm ss^{-1}}$ which is roughly consistent with the predictions from GR of $-8.2\pm1.7\times10^{-12}\,{\rm ss^{-1}}$ (see Fig. ~2). Since the object is very faint ($g=19.1\,{\rm mag}$), it required extensive observations over a 13 month period in order to get sufficient accuracy. This makes further follow-up difficult and limits the accuracy that can be reached. 

However, most recently the tight, eclipsing binary system ($CD-30,^{\circ}11223$, see Fig.~1, Geier et al. [2013], hereafter G13) has been discovered, which turned out to be much brighter ($V=11.9\,{\rm mag}$). $CD-30^{\circ}11223$ is composed of a hot subdwarf star (sdB) and a carbon/oxygen WD. It has a short period of only 0.048 days (1.2 hrs). In G13, the authors perform a detailed spectroscopic and photometric analysis of this object. Because $CD-30,^{\circ}11223$ is a single-lined system, plausible assumptions have to be made to find a unique solution.
G13 therefore derived the binary parameters using two different approaches. Firstly, they assume tidal synchronisation of the sdB primary. From this they arrive at a radius for the WD,  which is 10 $\%$ smaller than the predicted value from the zero-temperature mass-radius relation. In this case, the mass of the sdB is constrained to 0.47, the of the WD companion to 0.74$\subsun$. They obtain a slightly different set of parameters by restricting the WD to be within 2 $\%$ of the theoretical mass-radius relation. This leads to higher values of 0.54 and 0.79 M$\subsun$ for the masses of the sdB and the WD. Based on SWASP photometry gathered between 2006 and 2011, G13 derive an upper limit for the orbital decay of $-4.4\times10^{-12}\,{\rm ss^{-1}}$, only about one order of magnitude higher than the theoretically expected decay of $6\times10^{-13}\,{\rm ss^{-1}}$. The authors conclude that within a few years from the last observation, the orbital shrinkage will be detectable.  

Furthermore, the authors use the formalism given in Roelofs et al. [2007] to calculate gravitational wave strain $h$. They find a high log $h$ of -21.5 $\pm$ 0.3 and thus expect the object to be a strong source for detection of gravity waves using future space based missions like NGO/eLISA (Kilic et al. [2012]). Thus, this candidate will act as verification binary and offers a unique opportunity to anchor GW missions in the future, by comparing GW measurements to those in the optical wavelengths. Being a very bright object, $CD-30,^{\circ}11223$ is the perfect laboratory for studying such effects.}

\ImmediateObjective{Here were propose time-resolved photometry of $CD-30,^{\circ}11223$ with FORS2 to detect the shrinkage of the orbit by combining these data with existing photometry from the ASAS and SWASP planetary transit surveys and the SOAR telescope (G13) as described in Hermes et al. [2012] covering a total timebase of 15 years. Since the data from ASAS and SWASP were obtained with small telescopes, the achievable accuracy of the $\dot{p}$ measurement is limited. However, combined with only a few more high-precision eclipse times obtained with FORS2, we will be able to measure the cumulative shift of eclipse timing of $\sim28\,{\rm s}$ (calculated as outlined in Piro [2011]).
Furthermore, we want to resolve the discrepancy between the orbital parameters derived under the two separate assumptions. The different component masses should lead to subtle differences in the shape of the light curve, which is dominated by the sinusoidal signal originating from the ellipsoidal deformation of the sdB primary. Once we solved the discrepancy and determined the parameters of the system with high accuracy, $CD-30,^{\circ}11223$ will be a perfect verification binary for future space based gravitational wave missions like NGO/eLISA, in contrary to the semi-detached ultracompact binaries which have in general high uncertainties on the derived parameters. 
Light curves of highest accuracy are necessary to achieve both goals, measuring the orbital decay and derive system parameter with high accuracy. Available measurements with SOAR provide an SNR of $\sim200$, whereas observations with FORS2 can increase this value ten-fold to $\sim2000$. {\bf Due to its brightness, the observations can be conducted under unfavorable conditions.}} 

%
%---- THE SECOND PAGE OF THE SCIENCE CASE CAN INCLUDE FIGURES ----------
%
% Up to ONE page of figures can be added to your proposal.  
% The text and figures of the scientific description must not
% exceed TWO pages in total. 
% If you use color figures, do make sure that they are still readable
% if printed in black and white. Figures must be in PDF or JPEG format.
% Each figure has a size limit of 1MB.
% MakePicture and MakeCaption are optional macros and can be commented out.

\MakePicture{pfancy.pdf}{angle=0,width=10cm}
\MakeCaption{Fig.~1: Upper panel: V-band light curve of CD$-$30$^\circ$11223 taken with SOAR/Goodman (green) with superimposed model (red) plotted twice against the orbital phase for better visualisation. The dashed red curve marks the same model without transits and eclipses. The sinusoidal variation is caused by the ellipsoidal deformation of the hot subdwarf as a result of the tidal influence of the compact white dwarf. The difference in the maxima between phase $0.25$ and $0.75$ originates from the relativistic Doppler boosting effect, which is usually not detectable with ground-based telescopes. Lower panels: Close-up on the transit of the WD in front of the sdB (left). It is even possible to detect the eclipse of the WD by the sdB (right).}
\MakePicture{f3.pdf}{angle=0,width=10cm}
\MakeCaption{Fig.~2: $O-C$ diagrams of the orbital evolution in J0651 since April 2011; blue dots represent data from McDonald Observatory and APO, green squares from Gemini-North, and maroon triangles from GTC. The top panel shows the change in mid-eclipse times as determined by light curve modeling, and the best-fit parabola yields an estimate for the observed rate of orbital period change. Additionally, the bottom panel shows the results from a model-independent, linear least-squares fit using the orbital period and higher harmonics. The dotted line at $(O-C)=0$ shows the line of zero orbital decay, while the grey dashed line shows the predicted orbital decay expected solely from gravitational wave radiation. Using both methods, our early results match the GR prediction to the 1-$\sigma$ level.}



\MakeCaption{
  References:
  Geier, S., et al. 2013, A $\&$ A, 554, A54
  
  Brown, W. et al. 2011, ApJ, 737L, 23B
  
  Hermes et al. 2012, ApJ, 757L, 21H

  Kilic et al. 2012, ApJ, 751, 141
  
  Piro 2011, ApJ, 740, L53

  Roelofs et al. 2007, ApJ, 666, 1174
  

   }

%%%%%%%%%%%%%%%%%%%%%%%%%%%%%%%%%%%%%%%%%%%%%%%%%%%%%%%%%%%%%%%%%%%%%%
%%%%% THE PAGE OF TECHNICAL JUSTIFICATIONS %%%%%%%%%%%%%%%%%%%%%%%%%%%%%
%%%%%%%%%%%%%%%%%%%%%%%%%%%%%%%%%%%%%%%%%%%%%%%%%%%%%%%%%%%%%%%%%%%%%%%%
%
%---- BOX 8 ------------------------------------------------------------
%
% Provide below a careful justification of the requested lunar phase
% and of the requested number of nights or hours.  
% All macros in this box are NOT checked at the pdfLaTeX compilation.

\WhyLunarPhase{We do not require a specific lunar phase since our candidate object is bright and can be observed, even at 14 days after new moon. }  

\WhyNights{We calculate exposure times without a preference for lunar phase or sub-arcsecond seeing. The template spectrum is Pickles B2IV and the detector with the blue optimized CCD. For our required signal to noise of about 2000, we need single 2 second exposures with FORS2 in the B band. We will use the B-type star TYC 7281$-$1465$-$1 at a distance of 4.5 arcmin as comparison object. It has similar colours and a similar brightness of $V=11.6\,{\rm mag}$. Furthermore, a light curve obtained by the ASAS survey shows no variations. In this way we will be able to correct for airmass and other atmospheric variations.

In order to achieve an accurate measurement of the eclipse times, we request 1 hour of consecutive single exposures (matching one service mode block), which is sufficient to cover a significant part of the orbital period of 1.2 hours. In order to measure the $\dot{p}$, we require 4 such observations distributed over the semester.}

\TelescopeJustification{Available data for $CD-30^{\circ}11223$ include photometry from the ASAS and SWASP surveys taken from 2000 to 2011 and a short V-band light curve from the SOAR telescope taken in 2012. However, the precision on the period is not sufficient for a measurement of the $\dot{p}$. With FORS2 we will achieve the required precision in the eclipse times with only a few hours of observing time, because we will be able to clearly see the shallow eclipses ($\sim0.5\%$) in each of the high-quality light curves. This is not guaranteed, if we would use a smaller telescope. The SOAR light curve for example was taken under exceptionally good conditions. {\bf In contrast to that, due to the brightness of the object, the observation can be conducted under otherwise unfavorable conditions as a filler programme with FORS2.}}

\ModeJustification{Since this is only one target to be observed and is bright enough to be observed in sub-ideal weather conditions, we have applied for service mode observations}


% Please specify the type of calibrations needed.
% In case of special calibration the second parameter is used to enter 
% specific details.
% Valid values: standard, special
%\Calibrations{special}{Adopt a special calibration}
\Calibrations{standard}{}


%%%%%%%%%%%%%%%%%%%%%%%%%%%%%%%%%%%%%%%%%%%%%%%%%%%%%%%%%%%%%%%%%%%%%%%
%% PAGE OF BOXES 9-10  %%%%%%%%%%%%%%%%%%%%%%%%%%%%%%%%%%%%%%%%%%%%%%%%
%%%%%%%%%%%%%%%%%%%%%%%%%%%%%%%%%%%%%%%%%%%%%%%%%%%%%%%%%%%%%%%%%%%%%%%
%
%---- BOX 9 -- Use of ESO Facilities --------------------------------
%
% This macro is optional and can be commented out.
% It is also NOT checked at the pdfLaTeX compilation.
% LastObservationRemark: Report on the use of the ESO facilities during
%  the last 2 years (4 observing periods). Describe the status of the
%  data obtained and the scientific output generated.

\LastObservationRemark{
{\bf 090.D-0012:} Geier, Heber, Schaffenroth, et al. (Hot subdwarf stars with substellar companions): Spectroscopic follow-up of RV-variable subdwarfs, 75\% lost due to bad weather.\protect{\newline}
{\bf 091.D-0038:} Geier, Heber, Schaffenroth, et al. (Hot subdwarf stars with substellar companions): Spectroscopic follow-up of RV-variable subdwarfs, 90\% success. Paper in preparation.\protect{\newline}
{\bf 092.D-0040:} Geier, Heber, Schaffenroth, et al. (Hot subdwarf stars with substellar companions):  spectroscopic follow-up of RV-variable subdwarfs, 100\% success. Paper in preparation.\protect{\newline}
{\bf 093.D-0127:} Geier, Heber, Kupfer, et al. Hot subdwarfs at hypervelocity - donor remnants of double-detonation SN\,Ia?: time allocated in service mode, no observations yet
}

%
%---- BOX 9a -- ESO Archive ------------------------------------------
%
% Are the data requested in this proposal in the ESO Archive
% (http://archive.eso.org)? If yes, explain the need for new data.
% This macro is NOT checked at the pdfLaTeX compilation.

\RequestedDataRemark{No,the data for the object are not available in the ESO Archive}

%
%---- BOX 9b -- ESO GTO/Public Survey Programme Duplications---------
%
% If any of the targets/regions in ongoing GTO Programmes and/or
% Public Surveys are being duplicated here, please explain why.
% This macro is optional and can be commented out.
% It is also NOT checked at the pdfLaTeX compilation.

%\RequestedDuplicateRemark{}

%
%---- BOX 10 ------ Applicant(s) publications ---------------------
%
% Applicant's publications related to the subject of this proposal
% during the past two years.  Use the simplified abbreviations for
% references as in A&A.  Separate each reference with the following
% usual LaTex command: \smallskip\\
%   
%   Name1 A., Name2 B., 2001, ApJ, 518, 567: Title of article1
%   \smallskip\\
%   Name3 A., Name4 B., 2002, A\&A, 388, 17: Title of article2
%   \smallskip\\
%   Name5 A. et al., 2002, AJ, 118, 1567: Title of article3
%
% This macro is NOT checked at the pdfLaTeX compilation.

\Publications{
Barlow, B. N., Kilkenny, D., Drechsel, H., et al. 2013, MNRAS, 430, 22: EC 10246-2707 an eclipsing subdwarf
B + M dwarf binary

Geier, S. 2013, A\&A, 549, 110: Hot subdwarf stars in close-up view III. Metal abundances of subdwarf B stars

Geier, S., \& Heber, U. 2012, A\&A, 543, 149: Hot subdwarf stars in close-up view. II. Rotational properties of
single and wide binary subdwarf B stars

Geier, S., Heber, U., Edelmann, H., et al. 2013, A\&A, 557,112: Hot subdwarf stars in close-up view IV. Helium
abundances and the 3He isotopic anomaly of subdwarf B star

Geier, S., Heber, U., Heuser, C., et al. 2013, A\&A, 551, 4: The subdwarf B star SB 290 - A fast rotator on the
extreme horizontal branch

Geier, S., Marsh, T, R., Wang, B., et al. 2013, A\&A, 554, 54: A progenitor binary and an ejected mass donor
remnant of faint type Ia supernovae

Geier, S., Schaffenroth, V., Hirsch, H., et al. 2012, AN, 333, 431: MUCHFUSS - Massive Unseen Companions
to Hot Faint Underluminous Stars from SDSS

Geier, S., \O stensen, R., Heber, U., et al. 2014, A\&A,562,95:
Orbital solutions of eight close sdB binaries and constraints on the nature of the unseen companions

Kupfer, T., Groot, P. J., Levitan, D., et al. 2013, MNRAS, 432, 2048: Orbital periods and Accretion disc
structure of four AM CVn systems

\O stensen, R., Geier, S., Schaffenroth, V., et al. 2013, A\&A, 559, 350: Binaries discovered by the MUCHFUSS project. FBS 0117+396: An sdB+dM binary with a pulsating primary

Schaffenroth, V., Geier, S., Drechsel, H., et al. 2013, A\&A, 553, 18: A new bright eclipsing hot subdwarf binary
from the ASAS and SuperWASP surveys

Schaffenroth, V., Geier, S. Heber,U. et al. 2014, A\&A, in press: Binaries discovered by the MUCHFUSS project: SDSS J162256.66$+$473051.1 - An eclipsing subdwarf B binary with a brown dwarf companion
  
}

%%%%%%%%%%%%%%%%%%%%%%%%%%%%%%%%%%%%%%%%%%%%%%%%%%%%%%%%%%%%%%%%%%%%%%%%
%%%%% THE PAGE OF THE TARGET/FIELD LIST %%%%%%%%%%%%%%%%%%%%%%%%%%%%%%%%
%%%%%%%%%%%%%%%%%%%%%%%%%%%%%%%%%%%%%%%%%%%%%%%%%%%%%%%%%%%%%%%%%%%%%%%%
%
%---- BOX 11 -----------------------------------------------------------
%
% Complete list of targets/fields requested.  The macro takes nine
% parameters: run ID, target field/name, RA, Dec, time on target, magnitude, 
% diameter, additional information, reference star.
%
% 1. RUN ID
% Valid values: run IDs specified in BOX 3
%
% 2. TARGET FIELD/NAME
%
% 3. RA (J2000)
% Format: hh mm ss.f, or hh mm.f, or hh.f
% Use 00 00 00 for unknown coordinates
% This parameter is NOT checked at the pdfLaTeX compilation.
% 
% 4. Dec (J2000)
% Format: dd mm ss, or dd mm.f, or dd.f
% Use 00 00 00 for unknown coordinates
% This parameter is NOT checked at the pdfLaTeX compilation.
%
% 5. TIME ON TARGET
% Format: hours (overheads and calibration included)
% This parameter is NOT checked at the pdfLaTeX compilation.
%
% 6. MAGNITUDE
% This parameter is NOT checked at the pdfLaTeX compilation.
%
% 7. ANGULAR DIAMETER
% This parameter is NOT checked at the pdfLaTeX compilation.
%
% 8. ADDITIONAL INFORMATION
% Any relevant additional information may be inserted here.
% For APEX and CRIRES runs, the requested PWV upper limit MUST
% be specified for each target using this field.
% For APEX runs, the acceptable LST range MUST also be specified here.
% This parameter is NOT checked at the pdfLaTeX compilation.
%
% 9. REFERENCE STAR ID
% See Users' Manual.
% This parameter is NOT checked at the pdfLaTeX compilation.
%
% Long lists of targets will continue on the last page of the
% proposal.
%
%                       ** VERY IMPORTANT ** 
% The scheduling of your programme will take into account ALL targets
% given in this list. INCLUDE ONLY TARGETS REQUESTED FOR P94 !
% (except for VLT-XMM proposals)
%
% DO NOT USE ANY TEX/LATEX MACROS FOR THE TARGETS

%\Target{ABD}{NGC 104}{00 24 06}{-72 04 58}{3.0}{5}{30 min}{47 Tuc}{}
\Target{A}{$CD-30^{\circ}11223$}{14 11 16.2}{-30 53 03}{4h}{11.9}{}{}{}
%\Target{BC}{NGC 1851}{05 14 06.3}{-40 02 50}{8.0}{8.8}{}{glob. cluster}{}
%\Target{B}{NGC 1316}{03 22 41.5}{-37 12 33}{15.0}{9.7}{10 min}{Fornax  A}{}
%\Target{B}{NGC 1365}{03 33 36}{-36 08 27}{15.0}{10}{}{Seyfert gal.}{}
%\Target{C}{M 42}{05 35.3}{-05 23.5}{2.0}{4}{1 deg}{}{}
%\Target{C}{Rosette}{06 33.7}{+04 59.9}{3.0}{}{1 deg}{NGC 2237}{}
%\Target{D}{NGC 2997}{09 45 38}{-31 11 25}{10.0}{}{}{Sc galaxy}{S133231219553}
%\Target{E}{Alpha Ori}{06 45 08.9}{-16 42 58}{1}{-1.4}{6 mas}{Sirius}{}
%\Target{F}{Alpha Ori}{06 45 08.9}{-16 42 58}{1}{-1.4}{6 mas}{Sirius}{}


%                      ***************** 
%                      ** PWV limits **
% For CRIRES and all APEX instruments users must specify the PWV upper
% limits for each target. For example:
%\Target{}{Alpha Ori}{06 45 08.9}{-16 42 58}{1}{-1.4}{6 mas}{PWV=1.0mm, Sirius}{}
%\Target{}{HD 104237}{12 00 05.6}{-78 11 33}{1}{}{}{PWV<0.7mm;LST=9h00-15h00}{}
%
%                      *****************

% Use TargetNotes to include any comments that apply to several or all
% of your targets.
% This macro is NOT checked at the pdfLaTeX compilation.

%\TargetNotes{A note about the targets and/or strategy of selecting the targets during the run. For APEX runs please remember to specify the PWV limits for each target under 'Additional info' in the table above.}

%%%%%%%%%%%%%%%%%%%%%%%%%%%%%%%%%%%%%%%%%%%%%%%%%%%%%%%%%%%%%%%%%%%%%%%%
%%%%% TWO PAGES OF SCHEDULING REQUIREMENTS %%%%%%%%%%%%%%%%%%%%%%%%%%%%%
%%%%%%%%%%%%%%%%%%%%%%%%%%%%%%%%%%%%%%%%%%%%%%%%%%%%%%%%%%%%%%%%%%%%%%%%
%
%---- BOX 12 -----------------------------------------------------------
%

% Uncomment the following line if the proposal involves time-critical
% observations, or observations to be performed at specific time
% intervals. Please leave these brackets blank. Details of time
% constraints can be entered in Special Remarks and using the
% other flags in Box 13.
%
%
%\HasTimingConstraints{}

%
% The timing constraint macros listed below 
% are optional and can be commented out:
% \HasTimingConstraints, \RunSplitting, \Link and \TimeCritical
% They are also NOT checked at the pdfLaTeX compilation.


% 1. RUN SPLITTING, FOR A GIVEN ESO TELESCOPE (Visitor Mode only)
%
% 1st argument: run ID
% Valid values: run IDs specified in BOX 3
%
% 2nd argument: run splitting requested for sub-runs
% This parameter is NOT checked at the pdfLaTeX compilation.

%\RunSplitting{B}{1,10s,1}
%\RunSplitting{C}{2,10s,2,20w,2,15s,4H2}


% 2. LINK FOR COORDINATED OBSERVATIONS BETWEEN DIFFERENT RUNS.
%\Link{B}{after}{A}{10}
%\Link{C}{after}{B}{}
%\Link{E}{simultaneous}{F}{}

% 3. UNSUITABLE PERIOD(S) OF TIME
%
% 1st argument: run ID
% Valid values: run IDs specified in BOX 3
%
% 2nd argument: Chilean start date for the unsuitable time
% Format: dd-mmm-yyyy
% This parameter is NOT checked at the pdfLaTeX compilation.
%
% 3rd argument: Chilean end date for the unsuitable time
% Format: dd-mmm-yyyy
% This parameter is NOT checked at the pdfLaTeX compilation.

%\UnsuitableTimes{A}{15-jan-15}{18-jan-15}{Candidate is not visible during this part of the year}
%\UnsuitableTimes{B}{15-jan-15}{18-jan-15}{Insert explanation of unsuitable time here.}
%\UnsuitableTimes{C}{20-jan-15}{23-jan-15}{Insert explanation of unsuitable time here.}


%
%---- BOX 12 contd.. -- Scheduling Requirements 
%

% SPECIFIC DATE(S) FOR TIME-CRITICAL OBSERVATIONS
% Please note: The dates must correspond to the LOCAL CHILEAN observing dates.
%
% The 2nd and 3rd parameters are NOT checked at the pdfLaTeX compilation.
% 1st argument: run ID
% Valid values: run IDs specified in BOX 3
%
% 2nd argument: Chilean start date for the critical period.
% Format: dd-mmmm-yyyy 
%
% 3rd argument: Chilean end date for the critical period.
% Format: dd-mmmm-yyyy

%\TimeCritical{A}{12-nov-14}{14-nov-14}{Insert reason for time-critical observations.}
%\TimeCritical{D}{1-nov-14}{2-nov-14}{Insert reason for time-critical observations.}
%\TimeCritical{D}{17-nov-14}{18-nov-14}{Insert reason for time-critical observations.}
%\TimeCritical{D}{23-nov-14}{24-nov-14}{Insert reason for time-critical observations.}



%%%%%%%%%%%%%%%%%%%%%%%%%%%%%%%%%%%%%%%%%%%%%%%%%%%%%%%%%%%%%%%%%%%%%%%%
%
%---- BOX 14 -----------------------------------------------------------
%
% INSTRUMENT CONFIGURATIONS:
%
% Uncomment only the lines related to instrument configuration(s)
% needed for the acquisition of your planned observations. 
%
% 1st argument: run ID
% Valid values: run IDs specified in BOX 3
%
% 2nd argument: instrument
% This parameter is NOT checked at the pdfLaTeX compilation.
%
% 3rd argument: mode
% This parameter is NOT checked at the pdfLaTeX compilation.
%
% 4th argument: additional information
% This parameter is NOT checked at the pdfLaTeX compilation.
%
% All parameters are mandatory and cannot be empty. Do NOT specify
% Instrument Configurations for alternative runs.

% Examples (to be commented or deleted)


\INSconfig{A}{FORS2}{IMG}{B}

%
% Real list of instrument configurations

%%%%%%%%%%%%%%%%%%%%%%%%%%%%%%%%%%%%%%%%%%%%%%%%%%%%%%%%%%%%%%%%%%%%%%%%%
% Paranal
%
%
%-----------------------------------------------------------------------
%---- FORS2 at the VLT-UT1 (ANTU) --------------------------------------
%-----------------------------------------------------------------------
%If you require the E2V detector please select this option as well as
%the required mode below.
%\INSconfig{}{FORS2}{Detector}{E2V}
%
%If you require the MIT detector please select this option as well as
%the required mode below.
%\INSconfig{}{FORS2}{Detector}{MIT}
%
%\INSconfig{}{FORS2}{collimator}{HR}
%\INSconfig{}{FORS2}{PRE-IMG}{ESO filters: provide list HERE}
%\INSconfig{}{FORS2}{IMG}{ESO filters: provide list HERE}
%\INSconfig{}{FORS2}{IMG}{User's own filters (to be described in text)}
%\INSconfig{}{FORS2}{IPOL}{ESO filters: provide list HERE}
%\INSconfig{}{FORS2}{IPOL}{User's own filters (to be described in text)}
%\INSconfig{}{FORS2}{LSS}{Provide list of grism+filter combinations HERE}
%\INSconfig{}{FORS2}{MOS}{Provide list of grism+filter combinations HERE}
%\INSconfig{}{FORS2}{PMOS}{Provide list of grism+filter combinations HERE}
%\INSconfig{}{FORS2}{MXU}{Provide list of grism+filter combinations HERE}
%\INSconfig{}{FORS2}{HITI}{ESO filters: provide list HERE}
%\INSconfig{}{FORS2}{HIT-OS}{Provide list of grisms HERE}
%\INSconfig{}{FORS2}{HIT-MS}{Provide list of grisms HERE}
%\INSconfig{}{FORS2}{RRM}{yes}
%
%-----------------------------------------------------------------------
%---- KMOS at the VLT-UT1 (ANTU) ---------------------------------------
%-----------------------------------------------------------------------
%
%\INSconfig{}{KMOS}{IFU}{provide list of settings (IZ, YJ, H, K, HK) here} 
%
%-----------------------------------------------------------------------
%---- NAOS/CONICA at the VLT-UT1 (ANTU)  -------------------------------
%-----------------------------------------------------------------------
%
%\INSconfig{}{NACO}{PRE-IMG}{provide list of filters HERE}
%
% Specify the NGS name, distance from target and magnitude
%(Vmag preferred, otherwise Rmag) in the target list,
% and uncomment the following line
%\INSconfig{}{NACO}{NGS}{-}
%
%\INSconfig{}{NACO}{Special Cal}{Select if you have special calibrations}
%\INSconfig{}{NACO}{Pupil Track}{Select if you need pupil tracking mode}
%\INSconfig{}{NACO}{Cube}{Select if you need cube mode}
%
%\INSconfig{}{NACO}{SAM VIS-WFS}{Provide list of masks and filters HERE}
%\INSconfig{}{NACO}{SAM IR-WFS}{Provide list of masks and filters HERE}
%\INSconfig{}{NACO}{SAMPol VIS-WFS}{Provide list of masks and filters HERE}
%\INSconfig{}{NACO}{SAMPol IR-WFS}{Provide list of masks and filters HERE}
%
%\INSconfig{}{NACO}{IMG 54 mas/px IR-WFS}{provide list of filters HERE}
%\INSconfig{}{NACO}{IMG 27 mas/px IR-WFS}{provide list of filters HERE}
%\INSconfig{}{NACO}{IMG 13 mas/px IR-WFS}{provide list of filters HERE}
%\INSconfig{}{NACO}{IMG 54 mas/px VIS-WFS}{provide list of filters HERE}
%\INSconfig{}{NACO}{IMG 27 mas/px VIS-WFS}{provide list of filters HERE}
%\INSconfig{}{NACO}{IMG 13 mas/px VIS-WFS}{provide list of filters HERE}
%
%\INSconfig{}{NACO}{CORONA AGPM VIS-WFS}{provide list of filters (L',NB_3.74,NB_4.05) HERE}
%\INSconfig{}{NACO}{CORONA AGPM IR-WFS}{provide list of filters (L',NB_3.74,NB_4.05) HERE}
%
%\INSconfig{}{NACO}{POL 54 mas/px IR-WFS}{provide list of filters HERE}
%\INSconfig{}{NACO}{POL 27 mas/px IR-WFS}{provide list of filters HERE}
%\INSconfig{}{NACO}{POL 13 mas/px IR-WFS}{provide list of filters HERE}
%\INSconfig{}{NACO}{POL 54 mas/px VIS-WFS}{provide list of filters HERE}
%\INSconfig{}{NACO}{POL 27 mas/px VIS-WFS}{provide list of filters HERE}
%\INSconfig{}{NACO}{POL 13 mas/px VIS-WFS}{provide list of filters HERE}
%
%\INSconfig{}{NACO}{APP 54 mas/px IR-WFS}{select Lp and/or NB_4.05}
%\INSconfig{}{NACO}{APP 27 mas/px IR-WFS}{select Lp and/or NB_4.05}
%\INSconfig{}{NACO}{APP 54 mas/px VIS-WFS}{select Lp and/or NB_4.05}
%\INSconfig{}{NACO}{APP 27 mas/px VIS-WFS}{select Lp and/or NB_4.05}
%
%\INSconfig{}{NACO}{SPEC IR-WFS}{provide the list of spectroscopic modes HERE}
%\INSconfig{}{NACO}{SPEC VIS-WFS}{provide the list of spectroscopic modes HERE}
% %
%
%
%-----------------------------------------------------------------------
%---- FLAMES at the VLT-UT2 (KUEYEN) -----------------------------------
%-----------------------------------------------------------------------
%\INSconfig{}{FLAMES}{UVES}{Specify the UVES setup below}
%\INSconfig{}{FLAMES}{GIRAFFE-MEDUSA}{Specify the GIRAFFE setup below}
%\INSconfig{}{FLAMES}{GIRAFFE-IFU}{Specify the GIRAFFE setup below}
%\INSconfig{}{FLAMES}{GIRAFFE-ARGUS}{Specify the GIRAFFE setup below}
%\INSconfig{}{FLAMES}{Combined: UVES + GIRAFFE-MEDUSA}{Specify the UVES and
%GIRAFFE setups below}
%\INSconfig{}{FLAMES}{Combined: UVES + GIRAFFE-IFU}{Specify the UVES and
%GIRAFFE setups below}
%\INSconfig{}{FLAMES}{Combined: UVES + GIRAFFE-ARGUS}{Specify the UVES and
%GIRAFFe setups below}
%
%
% If you have selected UVES, either standalone or in combined mode,
% please specify the UVES standard setup(s) to be used
%\INSconfig{}{FLAMES}{UVES}{standard setup Red 520}
%\INSconfig{}{FLAMES}{UVES}{standard setup Red 580}
%\INSconfig{}{FLAMES}{UVES}{standard setup Red 580 + simultaneous calibration}
%\INSconfig{}{FLAMES}{UVES}{standard setup Red 860}
%
%\INSconfig{}{FLAMES}{GIRAFFE}{fast readout mode 625kHz VM only}
%
% If you have selected GIRAFFE, either standalone or in combined mode
% please specify the GIRAFFE standard setups(s) to be used
%\INSconfig{}{FLAMES}{GIRAFFE}{standard setup HR01 379.0}
%\INSconfig{}{FLAMES}{GIRAFFE}{standard setup HR02 395.8}
%\INSconfig{}{FLAMES}{GIRAFFE}{standard setup HR03 412.4}
%\INSconfig{}{FLAMES}{GIRAFFE}{standard setup HR04 429.7}
%\INSconfig{}{FLAMES}{GIRAFFE}{standard setup HR05 447.1 A}
%\INSconfig{}{FLAMES}{GIRAFFE}{standard setup HR05 447.1 B}
%\INSconfig{}{FLAMES}{GIRAFFE}{standard setup HR06 465.6}
%\INSconfig{}{FLAMES}{GIRAFFE}{standard setup HR07 484.5 A}
%\INSconfig{}{FLAMES}{GIRAFFE}{standard setup HR07 484.5 B}
%\INSconfig{}{FLAMES}{GIRAFFE}{standard setup HR08 504.8}
%\INSconfig{}{FLAMES}{GIRAFFE}{standard setup HR09 525.8 A}
%\INSconfig{}{FLAMES}{GIRAFFE}{standard setup HR09 525.8 B}
%\INSconfig{}{FLAMES}{GIRAFFE}{standard setup HR10 548.8}
%\INSconfig{}{FLAMES}{GIRAFFE}{standard setup HR11 572.8}
%\INSconfig{}{FLAMES}{GIRAFFE}{standard setup HR12 599.3}
%\INSconfig{}{FLAMES}{GIRAFFE}{standard setup HR13 627.3}
%\INSconfig{}{FLAMES}{GIRAFFE}{standard setup HR14 651.5 A}
%\INSconfig{}{FLAMES}{GIRAFFE}{standard setup HR14 651.5 B}
%\INSconfig{}{FLAMES}{GIRAFFE}{standard setup HR15 665.0}
%\INSconfig{}{FLAMES}{GIRAFFE}{standard setup HR15 679.7}
%\INSconfig{}{FLAMES}{GIRAFFE}{standard setup HR16 710.5}
%\INSconfig{}{FLAMES}{GIRAFFE}{standard setup HR17 737.0 A}
%\INSconfig{}{FLAMES}{GIRAFFE}{standard setup HR17 737.0 B}
%\INSconfig{}{FLAMES}{GIRAFFE}{standard setup HR18 769.1}
%\INSconfig{}{FLAMES}{GIRAFFE}{standard setup HR19 805.3 A}
%\INSconfig{}{FLAMES}{GIRAFFE}{standard setup HR19 805.3 B}
%\INSconfig{}{FLAMES}{GIRAFFE}{standard setup HR20 836.6 A}
%\INSconfig{}{FLAMES}{GIRAFFE}{standard setup HR20 836.6 B}
%\INSconfig{}{FLAMES}{GIRAFFE}{standard setup HR21 875.7}
%\INSconfig{}{FLAMES}{GIRAFFE}{standard setup HR22 920.5 A}
%\INSconfig{}{FLAMES}{GIRAFFE}{standard setup HR22 920.5 B}
%\INSconfig{}{FLAMES}{GIRAFFE}{standard setup LR01 385.7}
%\INSconfig{}{FLAMES}{GIRAFFE}{standard setup LR02 427.2}
%\INSconfig{}{FLAMES}{GIRAFFE}{standard setup LR03 479.7}
%\INSconfig{}{FLAMES}{GIRAFFE}{standard setup LR04 543.1}
%\INSconfig{}{FLAMES}{GIRAFFE}{standard setup LR05 614.2}
%\INSconfig{}{FLAMES}{GIRAFFE}{standard setup LR06 682.2}
%\INSconfig{}{FLAMES}{GIRAFFE}{standard setup LR07 773.4}
%\INSconfig{}{FLAMES}{GIRAFFE}{standard setup LR08 881.7}
%
%\INSconfig{}{FLAMES}{GIRAFFE}{fast readout mode 625kHz VM only}
%
%-----------------------------------------------------------------------
%---- X-SHOOTER at the VLT-UT2 (KUEYEN)
%-----------------------------------------------------------------------
%
%\INSconfig{}{XSHOOTER}{300-2500nm}{SLT}
%\INSconfig{}{XSHOOTER}{300-2500nm}{IFU}
%
% Slits (SLT only):
%
%UVB arm, available slits in arcsec: 0.5, 0.8, 1.0, 1.3, 1.6, 5.0
%VIS arm, available slits in arcsec: 0.4, 0.7, 0.9, 1.2, 1.5, 5.0 
%NIR arm, available slits in arcsec: 0.4, 0.6, 0.6JH, 0.9, 0.9JH, 1.2, 5.0
%  The 0.6JH and 0.9JH include a stray light K-band blocking filter
%  that allow sky limited studies in J and H bands.
%
%The slits for IFU  are fixed and do not need to be mentioned here.
%
% Replace SLIT_UVB, SLIT_VIS, SLIT_NIR with the choice of the slits:
%\INSconfig{}{XSHOOTER}{SLT}{SLIT_UVB,SLIT_VIS,SLIT_NIR}
%
% Detector readout mode:
%
% UVB and VIS arms: available readout modes and binning:
% 100k-1x1, 100k-1x2, 100k-2x2, 400k-1x1, 400k-1x2, 400k-2x2
% The NIR readout mode is fixed  to NDR.
%
%\INSconfig{}{XSHOOTER}{IFU}{readout UVB,readout VIS,readout NIR}
%\INSconfig{}{XSHOOTER}{SLT}{readout UVB,readout VIS,readout NIR}
%
% Imaging mode 
% replace 'list of filters' by the actual filters you wish to use among:
% U, B, V, R, I, Uprime, Gprime, Rprime, Iprime, Zprime
% Please note that the imaging mode can only be used in combination with slit or IFU observations
%\INSconfig{}{XSHOOTER}{IMG}{list of filters}
%
%\INSconfig{}{XSHOOTER}{RRM}{yes}
%
%-----------------------------------------------------------------------
%---- UVES at the VLT-UT2 (KUEYEN) -------------------------------------
%-----------------------------------------------------------------------
%
%\INSconfig{}{UVES}{BLUE}{Standard setting: 346}
%\INSconfig{}{UVES}{BLUE}{Standard setting: 437}
%\INSconfig{}{UVES}{BLUE}{Non-std setting: provide central wavelength  HERE}
%
%\INSconfig{}{UVES}{RED}{Standard setting: 520}
%\INSconfig{}{UVES}{RED}{Standard setting: 580}
%\INSconfig{}{UVES}{RED}{Standard setting: 600}
%\INSconfig{}{UVES}{RED}{Iodine cell standard setting: 600}
%\INSconfig{}{UVES}{RED}{Standard setting: 860}
%\INSconfig{}{UVES}{RED}{Non-std setting: provide central wavelength HERE}
%
%\INSconfig{}{UVES}{DIC-1}{Standard setting: 346+580}
%\INSconfig{}{UVES}{DIC-1}{Standard setting: 390+564}
%\INSconfig{}{UVES}{DIC-1}{Standard setting: 346+564}
%\INSconfig{}{UVES}{DIC-1}{Standard setting: 390+580}
%\INSconfig{}{UVES}{DIC-1}{Non-std setting: provide central wavelength HERE}
%
%\INSconfig{}{UVES}{DIC-2}{Standard setting: 437+860}
%\INSconfig{}{UVES}{DIC-2}{Standard setting: 346+860}
%\INSconfig{}{UVES}{DIC-2}{Standard setting: 390+860}
%
%\INSconfig{}{UVES}{DIC-2}{Standard setting: 437+760}
%\INSconfig{}{UVES}{DIC-2}{Standard setting: 346+760}
%\INSconfig{}{UVES}{DIC-2}{Standard setting: 390+760}
%\INSconfig{}{UVES}{DIC-2}{Non-std setting: provide central wavelength HERE}
%
%\INSconfig{}{UVES}{Field Derotation}{yes}
%\INSconfig{}{UVES}{Image slicer-1}{yes}
%\INSconfig{}{UVES}{Image slicer-2}{yes}
%\INSconfig{}{UVES}{Image slicer-3}{yes}
%\INSconfig{}{UVES}{Iodine cell}{yes}
%\INSconfig{}{UVES}{Longslit Filters}{Provide list of filters HERE}
%
%\INSconfig{}{UVES}{RRM}{yes}
%
%
%-----------------------------------------------------------------------
%---- VIMOS at the VLT-UT3 (MELIPAL) -----------------------------------
%-----------------------------------------------------------------------
%
%\INSconfig{}{VIMOS}{PRE-IMG}{ESO filters: enter the list of filters}
%\INSconfig{}{VIMOS}{IMG}{ESO filters: enter the list of filters}
%\INSconfig{}{VIMOS}{IFU 0.67"/fibre}{LR-Red}
%\INSconfig{}{VIMOS}{IFU 0.67"/fibre}{LR-Blue}
%\INSconfig{}{VIMOS}{IFU 0.67"/fibre}{MR}
%\INSconfig{}{VIMOS}{IFU 0.67"/fibre}{HR-Red}
%\INSconfig{}{VIMOS}{IFU 0.67"/fibre}{HR-Orange}
%\INSconfig{}{VIMOS}{IFU 0.67"/fibre}{HR-Blue}
%
%\INSconfig{}{VIMOS}{IFU 0.33"/fibre}{LR-Red}
%\INSconfig{}{VIMOS}{IFU 0.33"/fibre}{LR-Blue}
%\INSconfig{}{VIMOS}{IFU 0.33"/fibre}{MR}
%\INSconfig{}{VIMOS}{IFU 0.33"/fibre}{HR-Red}
%\INSconfig{}{VIMOS}{IFU 0.33"/fibre}{HR-Orange}
%\INSconfig{}{VIMOS}{IFU 0.33"/fibre}{HR-Blue}
%
%\INSconfig{}{VIMOS}{MOS-grisms}{LR-Red}
%\INSconfig{}{VIMOS}{MOS-grisms}{LR-Blue}
%\INSconfig{}{VIMOS}{MOS-grisms}{MR}
%\INSconfig{}{VIMOS}{MOS-grisms}{HR-Red}
%\INSconfig{}{VIMOS}{MOS-grisms}{HR-Orange}
%\INSconfig{}{VIMOS}{MOS-grisms}{HR-Blue}
%
%\INSconfig{}{VIMOS}{MOS-slits-targets}{0.6" < slit width < 1.4", targets:stellar}
%\INSconfig{}{VIMOS}{MOS-slits-targets}{0.6" < slit width < 1.4", targets:extended}
%\INSconfig{}{VIMOS}{MOS-slits-targets}{slit width > 1.4", targets:stellar}
%\INSconfig{}{VIMOS}{MOS-slits-targets}{slit width > 1.4", targets:extended}
%\INSconfig{}{VIMOS}{MOS-masks}{Enter here number of mask sets (1 set = 4 quadrants)}
%
%
%%-----------------------------------------------------------------------
%---- HAWKI at the VLT-UT4 (YEPUN) -----------------------------------
%-----------------------------------------------------------------------
%
%\INSconfig{}{HAWKI}{PRE-IMG}{provide list of filters (Y,J,H,Ks,CH4,BrG,H2,NB0984,NB1060,NB2090) HERE}
%\INSconfig{}{HAWKI}{IMG}{provide list of filters (Y,J,H,Ks,CH4,BrG,H2,NB0984,NB1060,NB2090) HERE}
%\INSconfig{}{HAWKI}{BURST}{Provide list of filters  (Y,J,H,Ks,CH4,BrG,H2,NB0984,NB1060,NB2090) HERE}
%\INSconfig{}{HAWKI}{FASTJITT}{Provide list of filters  (Y,J,H,Ks,CH4,BrG,H2,NB0984,NB1060,NB2090) HERE}
%\INSconfig{}{HAWKI}{RRM}{yes}
%
%-----------------------------------------------------------------------
%---- MUSE at the VLT-UT4 (YEPUN) -----------------------------------
%-----------------------------------------------------------------------
%
%\INSconfig{}{MUSE}{WFM-NOAO-N}{}
%\INSconfig{}{MUSE}{WFM-NOAO-E}{}
%
%-----------------------------------------------------------------------
%---- SINFONI at the VLT-UT4 (YEPUN) -----------------------------------
%-----------------------------------------------------------------------
%

%\INSconfig{}{SINFONI}{PRE-IMG}{provide list of setting(s) (J,H,K,H+K)}
%
%\INSconfig{}{SINFONI}{IFS 250mas/pix no-AO}{provide list of setting(s) (J,H,K,H+K) HERE}
%\INSconfig{}{SINFONI}{IFS 100mas/pix no-AO}{provide list of setting(s) (J,H,K,H+K) HERE}
%
% If you plan to use a NGS, please specify the NGS name and magnitude (Rmag preferred,
% otherwise Vmag) in target list.
%\INSconfig{}{SINFONI}{IFS 250mas/pix NGS}{provide list of setting(s) (J,H,K,H+K) HERE}
%\INSconfig{}{SINFONI}{IFS 100mas/pix NGS}{provide list of setting(s) (J,H,K,H+K) HERE}
%\INSconfig{}{SINFONI}{IFS 25mas/pix NGS}{provide list of setting(s) (J,H,K,H+K) HERE}
%
% If you plan to use the LGS, please specify the TTS name and magnitude (Rmag preferred,
% otherwise Vmag) in target list.
%\INSconfig{}{SINFONI}{IFS 250mas/pix LGS}{provide list of setting(s) (J,H,K,H+K) HERE}
%\INSconfig{}{SINFONI}{IFS 100mas/pix LGS}{provide list of setting(s) (J,H,K,H+K) HERE}
%\INSconfig{}{SINFONI}{IFS 25mas/pix LGS}{provide list of setting(s) (J,H,K,H+K) HERE}
%
% If you plan to use the LGS without a TTS (seeing enhancer mode) then
% please leave the TTS name blank in the target list.
%\INSconfig{}{SINFONI}{IFS 250mas/pix LGS-noTTS}{provide list of setting(s) (J,H,K,H+K) HERE}
%\INSconfig{}{SINFONI}{IFS 100mas/pix LGS-noTTS}{provide list of setting(s) (J,H,K,H+K) HERE}
%\INSconfig{}{SINFONI}{IFS 25mas/pix LGS-noTTS}{provide list of setting(s) (J,H,K,H+K) HERE}
%
% Select if you have special calibrations
%\INSconfig{}{SINFONI}{Special Cal}{}
%
% Select if you need pupil tracking mode
%\INSconfig{}{SINFONI}{Pupil Track}{}
%
% Select for RRM
%\INSconfig{}{SINFONI}{RRM}{yes}
%
%
%-----------------------------------------------------------------------
%---- Interferometric Instruments --------------------------------------
%-----------------------------------------------------------------------
%
%-----------------------------------------------------------------------
%---- MIDI -------------------------------------------------------------
%-----------------------------------------------------------------------
%
%
%\INSconfig{}{MIDI}{PRISM}{CORR-FLUX}
%
% For MIDI + PRIMA - FSU
%\INSconfig{}{MIDI}{PRISM}{CORR-FLUX-F} 
%
%\INSconfig{}{MIDI}{PRISM}{HIGH-SENS}
%\INSconfig{}{MIDI}{GRISM}{HIGH-SENS}
%
%\INSconfig{}{MIDI}{PRISM}{SCI-PHOT}
%\INSconfig{}{MIDI}{GRISM}{SCI-PHOT}
%
%
%
%-----------------------------------------------------------------------
%---- AMBER ------------------------------------------------------------
%-----------------------------------------------------------------------
%
%\INSconfig{}{AMBER}{LR-HK-F}{2.2}
%\INSconfig{}{AMBER}{LR-HK}{2.2}
%\INSconfig{}{AMBER}{MR-K-F}{2.1}
%\INSconfig{}{AMBER}{MR-K}{2.1}
%\INSconfig{}{AMBER}{MR-H-F}{1.65} % << updated in P87
%\INSconfig{}{AMBER}{MR-H}{1.65}   % << updated in P87
%\INSconfig{}{AMBER}{MR-K-F}{2.3}
%\INSconfig{}{AMBER}{MR-K}{2.3}
%\INSconfig{}{AMBER}{HR-K}{Central wavelength selected from the list:
% 1.97929,2.01786,2.05643,2.09500,2.13357,2.17214,2.21071,2.24929,2.28786,2.32643,
% 2.36500,2.40357,2.44214,2.48071}
%\INSconfig{}{AMBER}{HR-K-F}{Central wavelength selected from the list:
% 1.97929,2.01786,2.05643,2.09500,2.13357,2.17214,2.21071,2.24929,2.28786,2.32643,
% 2.36500,2.40357,2.44214,2.48071}
% 
%where *-F means with FINITO
%
%-----------------------------------------------------------------------
%---- VIRCAM at VISTA --------------------------------------------------
%-----------------------------------------------------------------------
%
%\INSconfig{}{VIRCAM}{IMG}{provide list of filters here}
%
%-----------------------------------------------------------------------
%---- OMEGACAM at VST --------------------------------------------------
% This instrument is only available for GTO and Chilean programmes.
%-----------------------------------------------------------------------
%
%\INSconfig{}{OMEGACAM}{IMG}{provide list of filters here}
%
%%%%%%%%%%%%%%%%%%%%%%%%%%%%%%%%%%%%%%%%%%%%%%%%%%%%%%%%%%%%%%%%%%%%%%%%
% La Silla
%-----------------------------------------------------------------------
%---- EFOSC2 (or SOFOSC) at the NTT ------------------------------------
%-----------------------------------------------------------------------
%
%\INSconfig{}{EFOSC2}{PRE-IMG}{EFOSC2 filters: provide list here}
%\INSconfig{}{EFOSC2}{Imaging-filters}{EFOSC2 filters:  provide list here}
%\INSconfig{}{EFOSC2}{Imaging-filters}{ESO non EFOSC filters: provide ESOfilt No}
%\INSconfig{}{EFOSC2}{Imaging-filters}{User's own filters (to be described in text)}
%\INSconfig{}{EFOSC2}{Spectro-long-slit}{Grism\#1:320-1090}
%\INSconfig{}{EFOSC2}{Spectro-long-slit}{Grism\#2:510-1100}
%\INSconfig{}{EFOSC2}{Spectro-long-slit}{Grism\#3:305-610}
%\INSconfig{}{EFOSC2}{Spectro-long-slit}{Grism\#4:409-752}
%\INSconfig{}{EFOSC2}{Spectro-long-slit}{Grism\#5:520-935}
%\INSconfig{}{EFOSC2}{Spectro-long-slit}{Grism\#6:386-807}
%\INSconfig{}{EFOSC2}{Spectro-long-slit}{Grism\#7:327-524}
%\INSconfig{}{EFOSC2}{Spectro-long-slit}{Grism\#8:432-636}
%\INSconfig{}{EFOSC2}{Spectro-long-slit}{Grism\#11:338-752}
%\INSconfig{}{EFOSC2}{Spectro-long-slit}{Grism\#13:369-932}
%\INSconfig{}{EFOSC2}{Spectro-long-slit}{Grism\#14:310-509}
%\INSconfig{}{EFOSC2}{Spectro-long-slit}{Grism\#16:602-1032}
%\INSconfig{}{EFOSC2}{Spectro-long-slit}{Grism\#17:689-876}
%\INSconfig{}{EFOSC2}{Spectro-long-slit}{Grism\#18:470-677}
%\INSconfig{}{EFOSC2}{Spectro-long-slit}{Grism\#19:440-510}
%\INSconfig{}{EFOSC2}{Spectro-long-slit}{Grism\#20:605:715}
%\INSconfig{}{EFOSC2}{Spectro-long-slit}{Aperture: 0.5'', ... ,10.0''}
%
%\INSconfig{}{EFOSC2}{Spectro-long-slit}{Aperture: Shiftable}
%\INSconfig{}{EFOSC2}{Spectro-MOS}{PunchHead=0.95''}
%\INSconfig{}{EFOSC2}{Spectro-MOS}{PunchHead=1.12''}
%\INSconfig{}{EFOSC2}{Spectro-MOS}{PunchHead=1.45''}
%\INSconfig{}{EFOSC2}{Polarimetry}{$\lambda / 2$ retarder plate}
%\INSconfig{}{EFOSC2}{Polarimetry}{$\lambda / 4$ retarder plate}
%\INSconfig{}{EFOSC2}{Coronograph}{yes}
%
%
%-----------------------------------------------------------------------
%---- SOFI (or SOFOSC) at the NTT --------------------------------------------------
%-----------------------------------------------------------------------
%
%\INSconfig{}{SOFI}{PRE-IMG-LargeField}{Provide list of filters HERE}
%\INSconfig{}{SOFI}{Imaging-LargeField}{Provide list of filters HERE}
%\INSconfig{}{SOFI}{Burst}{Provide list of filters HERE}
%\INSconfig{}{SOFI}{FastPhot}{Provide list of filters HERE}
%\INSconfig{}{SOFI}{Polarimetry}{Provide list of filters HERE}
%\INSconfig{}{SOFI}{Spectroscopy-long-slit}{Blue Grism, Provide list of slits HERE}
%\INSconfig{}{SOFI}{Spectroscopy-long-slit}{Red Grism, Provide list of slits HERE}
%\INSconfig{}{SOFI}{Spectroscopy-high-res}{H, Provide list of slits HERE}
%\INSconfig{}{SOFI}{Spectroscopy-high-res}{K, Provide list of slits HERE}
%
%
%-----------------------------------------------------------------------
%---- HARPS at the 3.6 -------------------------------------------------
%-----------------------------------------------------------------------
%
%\INSconfig{}{HARPS}{spectro-Thosimult}{HARPS}
%\INSconfig{}{HARPS}{WAVE}{HARPS}
%\INSconfig{}{HARPS}{spectro-ObjA(B)}{HARPS}
%\INSconfig{}{HARPS}{spectro-ObjA(B)}{EGGS}
%\INSconfig{}{HARPS}{spectro-polarimetry}{linear}
%\INSconfig{}{HARPS}{spectro-polarimetry}{circular}
%
%
%%%%%%%%%%%%%%%%%%%%%%%%%%%%%%%%%%%%%%%%%%%%%%%%%%%%%%%%%%%%%%%%%%%%%%%%
% Chajnantor
%-----------------------------------------------------------------------
%---- SHFI at APEX ----------------------------------------------
%-----------------------------------------------------------------------
%
%\INSconfig{}{SHFI}{APEX-1}{Please enter Central Frequency 211 to 275 GHz}
%\INSconfig{}{SHFI}{APEX-2}{Please enter Central Frequency 275 to 370 GHz}
%\INSconfig{}{SHFI}{APEX-3}{Please enter Central Frequency 385 to 500 GHz} 
%\INSconfig{}{SHFI}{APEX-T2}{Please enter Central Frequency 1.25 to 1.39 THz}
%
%-----------------------------------------------------------------------
%---- LABOCA at APEX ----------------------------------------------
%-----------------------------------------------------------------------
%
%\INSconfig{}{LABOCA}{IMG}{-}
%\INSconfig{}{LABOCA}{PHOT}{-}
%
%-----------------------------------------------------------------------
%---- Artemis at APEX ----------------------------------------------
%-----------------------------------------------------------------------
%\INSconfig{}{ARTEMIS}{IMG}{350 um}
%-----------------------------------------------------------------------
%---- Supercam at APEX ----------------------------------------------
%-----------------------------------------------------------------------
%\INSconfig{}{SUPERCAM}{RECEIVER}{Enter frequency 329-360 GHz}
%
%-----------------------------------------------------------------------
%---- FLASH at APEX ----------------------------------------------
%-----------------------------------------------------------------------
%
%\INSconfig{}{FLASH}{-}{Please enter Central Frequency 272 to 377 GHz and 385 to 495 GHz}
%
%-----------------------------------------------------------------------
%---- CHAMP+ at APEX ----------------------------------------------
%-----------------------------------------------------------------------
%
%\INSconfig{}{CHAMPP}{-}{Please enter Central Frequency 620 to 729 GHz and 780 to 900 GHz}
%-----------------------------------------------------------------------




%%%%%%%%%%%%%%%%%%%%%%%%%%%%%%%%%%%%%%%%%%%%%%%%%%%%%%%%%%%%%%%%%%%%%%%%
%%%%% Interferometry PAGE %%%%%%%%%%%%%%%%%%%%%%%%%%%%%%%%%%%%%%%%%%%%%%
%%%%%%%%%%%%%%%%%%%%%%%%%%%%%%%%%%%%%%%%%%%%%%%%%%%%%%%%%%%%%%%%%%%%%%%%
%
% The \VLTITarget macro is only needed when requesting
% Interferometry, in which case it is MANDATORY to uncomment it and
% fill in the information. It takes the following parameters:
%
% 1st argument: run ID
% Valid values: run IDs specified in BOX 3
%
% 2nd argument: target name
% This parameter is NOT checked at the pdfLaTeX compilation.
%
% 3rd argument: visual magnitude
% Values with up to decimal places are allowed here.
% This parameter is NOT checked at the pdfLaTeX compilation.
%
% 4th argument: magnitude at wavelength of observation
% Values with up to decimal places are allowed here.
% This parameter is NOT checked at the pdfLaTeX compilation.
%
% 5th argument: wavelength of observation (in microns)
% Values with up to decimal places are allowed here.
% This parameter is NOT checked at the pdfLaTeX compilation.
%
% 6th argument: size at wavelength of observation (in mas)
% This parameter is NOT checked at the pdfLaTeX compilation.
%
% 7th argument: baseline
% UT observations are scheduled in terms of 3-telescope 
% baselines for AMBER and 2-telescope baselines for MIDI.
% For UT observations please specify one of the four available 
% AMBER baselines or one of the six available MIDI baselines.
%
% AT observations are scheduled in terms of 4-telescope 
% configurations (quadruplets). For these observations, the 
% time can be split among the different 3-telescope baselines 
% (for AMBER) or 2-telescope baselines (for MIDI); the exact 
% baselines will be specified at Phase 2.
% For AT observations, please specify only one of the 3 
% available AT quadruplets at this stage.
%
%
% 8th parameter: visibility for the specified configuration
% (at preferred hour angle or hour angle 0)
% This parameter is NOT checked at the pdfLaTeX compilation.
%
% For AMBER observations, please specify the three visibility
% values corresponding to the three baselines of the chosen 
% VLTI configurations, separated by "/"; up to two of these 
% values may be replaced by '*'.
% This parameter is NOT checked at the pdfLaTeX compilation.
%
% For AT observations, please use one typical baseline of the 
% quadruplet that you have specified in order to compute
% typical visibility values.
%
%
% 9th parameter: correlated magnitude
% (for the visibility values specified in the 8th parameter)
% This parameter is NOT checked at the pdfLaTeX compilation.
%
% 10th parameter: time on target in hours
% Values with up to decimal places are allowed here.
% This parameter is NOT checked at the pdfLaTeX compilation.
%
% Note: For MIDI observations in any mode, please indicate 10.6 as
% wavelength of observation.
%
% The available baselines for Period 94 are shown below.
% For AT observations, the time can be split at Phase 2 among the
% different 2-telescope (for MIDI) or 3-telescope (AMBER) baselines
% of the chosen quadruplet. All possible 2-telescope (MIDI) or
% 3-telescope(AMBER) baselines available in these quadruplets
% are offered in both service and visitor mode.
%
% 
% AMBER
% A1-B2-C1-D0
% A1-G1-K0-J3
% D0-H0-G1-I1
% UT1-UT2-UT3
% UT1-UT2-UT4
% UT1-UT3-UT4
% UT2-UT3-UT4
% 
% MIDI
% A1-B2-C1-D0
% A1-G1-K0-J3
% D0-H0-G1-I1
% UT1-UT2-57m
% UT1-UT3-102m
% UT1-UT4-130m
% UT2-UT3-47m
% UT2-UT4-89m
% UT3-UT4-62m
% 
% SpecialVLTI
% A1-B2-C1-D0
% A1-G1-K0-J3
% D0-H0-G1-I1
% UT1-UT2
% UT1-UT2-UT3
% UT1-UT2-UT3-UT4
% UT1-UT2-UT4
% UT1-UT3
% UT1-UT3-UT4
% UT1-UT4
% UT2-UT3
% UT2-UT3-UT4
% UT2-UT4
% UT3-UT4
% 

%\VLTITarget{E}{Alpha Ori}{-1.4}{-1.4}{10.6}{6}{UT1-UT2-UT3}{0.45/0.60/0.10}{0.3/-0.2/4.0}{2}
%\VLTITarget{F}{Alpha Ori}{-1.4}{-1.4}{10.6}{6}{D0-H0-G1-I1}{0.80}{-0.9}{1}

% You can specify here a note applying to all or some of your VLTI
% targets.  You should take advantage of this note to indicate
% suitable alternative baselines for your observations.
% This macro is NOT checked at the pdfLaTeX compilation.

%\VLTITargetNotes{Note about the VLTI targets, e.g., Run E can also be carried out using UT1-UT3-UT4.}


%%%%%%%%%%%%%%%%%%%%%%%%%%%%%%%%%%%%%%%%%%%%%%%%%%%%%%%%%%%%%%%%%%%%%%%%
%%%%% ToO PAGE %%%%%%%%%%%%%%%%%%%%%%%%%%%%%%%%%%%%%%%%%%%%%%%%%%%%%%%%%
%%%%%%%%%%%%%%%%%%%%%%%%%%%%%%%%%%%%%%%%%%%%%%%%%%%%%%%%%%%%%%%%%%%%%%%%
%
% The \ToOrun macro is needed only when requesting Target of
% Opportunity (ToO) observations, in which case it is MANDATORY to
% uncomment it and fill in the information. It takes the following
% parameters: 
%
% 1st argument: run ID
% Valid values: run IDs specified in BOX 3
%
% 2nd argument: nature of observation
% Valid values: RRM, ToO-hard, ToO-soft
%
% 3rd argument: number of targets per run
% This parameter is NOT checked at the pdfLaTeX compilation.
%
% 4th argument: number of triggers per targets
% (for RRM and ToO observations only)
% This parameter is NOT checked at the pdfLaTeX compilation.

%\TOORun{A}{RRM}{2}{3}
%\TOORun{B}{ToO-hard}{3}{1}

% You have the opportunity to add notes to the ToO runs by using
% the \TOONotes macro.
% This macro is NOT checked at the pdfLaTeX compilation.

%\TOONotes{Use this macro to add a note to the ToO page.}


%%%%%%%%%%%%%%%%%%%%%%%%%%%%%%%%%%%%%%%%%%%%%%%%%%%%%%%%%%%%%%%%%%%%%%%%
%%%%% VISITOR SPECIAL INSTRUMENT PAGE %%%%%%%%%%%%%%%%%%%%%%%%%%%%%%%%%%
%%%%%%%%%%%%%%%%%%%%%%%%%%%%%%%%%%%%%%%%%%%%%%%%%%%%%%%%%%%%%%%%%%%%%%%%
%
% The following commands are only needed when bringing a Visitor
% Special Instrument, in which case it is MANDATORY to uncomment them
% and provide all the required information.
%
%\Desc{}   %Description of the instrument and its operation
%\Comm{}   %On which telescope(s) has instrument been commissioned/used
%\WV{}     %Total weight and value of equipment to be shipped
%\Wfocus{} %Weight at the focus (including ancillary equipment)
%\Interf{} %Compatibility of attachment interface with required focus
%\Focal{}  %Back focal distance value
%\Acqu{}   %Acquisition, focusing, and guiding procedure
%\Softw{}  %Compatibility with ESO software standards (data handling)
%\Suppl{}  %Estimate of services expected from ESO (in person days)

%%%%%%%%%%%%%%%%%%%%%%%%%%%%%%%%%%%%%%%%%%%%%%%%%%%%%%%%%%%%%%%%%%%%%%%%
%%%%% THE END %%%%%%%%%%%%%%%%%%%%%%%%%%%%%%%%%%%%%%%%%%%%%%%%%%%%%%%%%%
%%%%%%%%%%%%%%%%%%%%%%%%%%%%%%%%%%%%%%%%%%%%%%%%%%%%%%%%%%%%%%%%%%%%%%%%
\MakeProposal
\end{document}


