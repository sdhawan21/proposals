\documentcass{article}
\begin{document}
\title{Near Infrared Spectroscopy of SN2014J at late epochs}
\maketitle
Late time Near Infrared spectra of Type Ia supernovae offer a unique window into the physics of the explosions. Due to their faintness, very few objects have been spectroscopically followed up at epochs close to a year after maximum light. A very small fraction of those have spectra at more than one epoch.
Getting a time series of spectra at intervals of $\sim$ 50 days 

\textbf{Iron Mass}:
The Fe II lines in the NIR spectra at late times allow for an independent estimate of the iron mass. In Spyromilio et al. [2004], the authors show that the Fe II emission is consistent with the $Fe^{+}$  mass from other methods, eg. 
the evolution of the NIR spectrum provides  direct evidence for Co to Fe decay. 

\textbf{Line Velocities}:
In Spyromilio et al. [2004], the authors note that the line velocities for their two spectra at +250 and +344 days are not the same, indicating a dependence of the nebular velocity on the epoch of observation. The nebular velocities have been used in studies to indicate a relation between the kinetic energy of the ejecta and the total energy from the $^{56} Ni$ decay (eg. Mazzali et al. [1998], Blondin et al. [2012], Silverman et al. [2013]). If there is an evolution in the line velocities then the late-time expansion velocity would be dependent on the exact epoch of measurement and hence, the results would need to be reconsidered. 

In order to discern convincingly whether there is an evolution of the expansion velocity with time, we would benefit greatly from a time-series at late times of SN2014J.  

\textbf{Extinction}:
\end{document}